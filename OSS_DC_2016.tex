%%%%%%%%%%%%%%%%%%%%%%%%%% author.tex %%%%%%%%%%%%%%%%%%%%%%%%%
%
% sample root file for your contribution to an IFIP volume
% published at Springer
%
% Use this file as a template for your own input.
%
%%%%%%%%%%%%%%%%%%%%%%%% Springer-Verlag %%%%%%%%%%%%%%%%%%%%%%%%%%


% RECOMMENDED %%%%%%%%%%%%%%%%%%%%%%%%%%%%%%%%%%%%%%%%%%%%%%%%%%%
\documentclass[ifip]{svmult}

% choose options for [] as required from the list
% in the Reference Guide, Sect. 2.2
\usepackage{epsfig} 
\usepackage{makeidx}         % allows index generation
\usepackage{graphicx}        % standard LaTeX graphics tool
                             % when including figure files
\usepackage{multicol}        % used for the two-column index
\usepackage[bottom]{footmisc}% places footnotes at page bottom
% etc.
% see the list of further useful packages
% in the Reference Guide, Sects. 2.3, 3.1-3.3

\makeindex             % used for the subject index
                       % please use the style sprmidx.sty with
                       % your makeindex program


%%%%%%%%%%%%%%%%%%%%%%%%%%%%%%%%%%%%%%%%%%%%%%%%%%%%%%%%%%%%%%%%%%%%%

\begin{document}

\title*{A Quantitative Analysis of Performance of the Key Parameter in Code Reviews:  Individuation of Defects}
%code review in software development in cloud-based system }
% Use \titlerunning{Short Title} for an abbreviated version of
% your contribution title if the original one is too long
\author{Dorealda Dalipaj\inst{1}}
% Use \authorrunning{Short Title} for an abbreviated version of
% your contribution title if the original one is too long
\institute{Universidad Rey Juan Carlos, Madrid \\
\texttt{dorealda.dalipaj@urjc.es}}
%
% Use the package "url.sty" to avoid
% problems with special characters
% used in your e-mail or web address
%
\maketitle

\textbf{\textit{Abstract -}} Code review is a well-established way to find defects, employed both in industrial and open source contexts, known to be cost-effective. 
Nowadays code review is tool assisted, in general by means of bug tracking systems that were developed to guide maintenance activities of 
software developers. Finding and removing defects close to their point of injection remains the main motivation and a key parameter
for review. Many developers hint that this practice results to be time efficient and more lightweight compared to the early software inspections 
performed in the 1980s. 
However, different studies (\cite{contribution1},~\cite{contribution9},~\cite{contribution20}) have shown that code review is not performing as expected. 
They argue that the performance of code review 
is low and that the actual outcome of code reviews in finding errors is less than the expected one. Furthermore, another study~\cite{contribution10} argues that many software programs 
rely on bug reports to correct software errors during maintenance, making developers spend too much time in managing and handling these bug reports. 
The aim of this study is to bring evidence by quantifying the performance of the very key parameter of code review: individuation of bugs.
We focus our case study on a cloud computing software, OpenStack~\cite{contribution11}. 
A developer can contribute to its source code repository using the Launchpad Bugtracking~\cite{contribution12} and the Gerrit Code Review~\cite{contribution13} tools.
The process by which we want to proceed is as follows. First, from Launchpad, we individuate the bug reports that are actually describing a defect, 
and extract the needed information to bring statistical evidence on the number of defects that are individuated in OpenStack through the 
years. Second, we compare the first results to the total number of issues reported, grouped by their different types, 
to have a proof of the rate at which the defects are discovered. And last we bring empirical evidence on the effort spent by developers, 
by means of time, to carry out the defect individuation process.

\section{Introduction}
\label{sec:1}
% Always give a unique label
% and use \ref{<label>} for cross-references
% and \cite{<label>} for bibliographic references
% use \sectionmark{}
% to alter or adjust the section heading in the running head

Code review, sometimes referred as peer review, is an activity in which people, other than the author of a code snippet, 
examine it for defects and improvement opportunities.
Code review is characterized \textit{as a systematic approach to examine a product in detail, 
using a predefined sequence of steps to determine if the product is fit for its intended use}~\cite{contribution14}. It is considered one of the most powerful software quality tools available. 


There have been different ways of performing defect detection since its beginning up to nowadays. The formal review or inspection 
according to Fanagan's approach~\cite{contribution15} required the conduction of an inspection meeting for actually finding defects. Different controlled experiments showed that there were no significant differences in the total number of defects found 
when comparing meeting-based with meetingless-based inspections~\cite{contribution17, contribution18}. Other 
studies~\cite{contribution19} were carried out and proved that more defects were identified with meetingless-based approaches. 
As a result a wide range of mechanisms and techniques of code review were developed. From static analysis~\cite{contribution3, contribution4, contribution5}, which examines the code in 
the absence of input data and without running the code and is tool based, to modern code review~ 
\cite{contribution6. contribution7, contribution8}, which aligned with the distributed 
nature of many projects is asynchronous and frequently supporting geographically distributed reviewers. Because of their many uses 
and benefits, code reviews are a standard part of the modern software engineering workflow.

It is generally accepted that quality in software remains a challenge due to defects presence. A major quality issue with 
software is that defects are a byproduct of the complex development process and the ability to develop software free of defects 
remains a big challenge for the software community. 
It is possible to improve the quality of a software product by injecting fewer defects or by identifying and 
removing defects injected. 

It is also generally accepted that the performance of software reviews is affected by several factors of the 
defect detection process. So, code review performance is associated with the effort spent to carry out the process and the number of 
defects found. 

A wide set of empirical studies and new approaches have been proposed to understand and improve the review process since it 
was introduced by Fagan~\cite{contribution15}. Sources of process variability range from structure (how the inspection is organized in several steps), inspection inputs (reviewer ability and product quality), 
techniques applied to defect identification that define how each step is carried out), 
context and tool support~\cite{contribution16}. Most empirical studies try to assess the impact of specific process settings on performance. 
A controlled experiment by Johnson and Tjahjono \cite{contribution17} showed that \textit{total defects identified, effort spent in the 
process, false positive defects, duplicates} are fundamental variables to analyse when controlling the performance of code review. 

\section{Discussion}
\label{sec:2}


%Static analysis examines code in 
%the absence of input data and without running the code, and can
%detect potential security violations (e.g., SQL injection), runtime errors (e.g., dereferencing a null pointer) and logical
%inconsistencies (e.g., a conditional test that cannot possibly be true). A growing number of projects and companies are concerned about 
%individuating bugs in their code and they are making this process part of their standard build and testing system. 
%According to \cite{contribution2}, Google has incorporated FindBugs into their standard testing and code review process, 
%and has fixed more than 1,000 issues in their internal code base identified by the tool.
%More powerful tools \cite{contribution3}, \cite{contribution4}, \cite{contribution5}, evaluate software in the
%abstract, without executing them or considering a specific input. It has become fairly clear that static analysis
%tools can find important defects in software.

Although code review is used in software engineering primarily for finding defects, several studies argue that they often 
do not find functionality defects.

Microsoft develops software in diverse domains, from high-end server enterprise data management solutions such
as SQL Server to mobile phone applications and smart phone apps to search engines. Over the past two years, 
a common tool for code review, \texttt{CodeFlow}, has achieved wide-spread adoption at Microsoft. 
The functionality of \texttt{CodeFlow} is similar to other review tools such \texttt{Mondrian}~ \cite{contribution6} (adopted at Google), \texttt{Phabricator}~\cite{contribution7} (adopted at Facebook) or the open-source \texttt{Gerrit}~\cite{contribution8}.

Two studies have been conducted at Microsoft on code review process. The first study took place with professional developers, testers, and managers at Microsoft~\cite{contribution9}. 
The results show that, although the top motivation driving code reviews is finding defects, the practice and the 
actual outcomes are less about finding errors than expected: Defect related comments comprise a small proportion, only 14\%, 
and mainly cover small logical low-level issues.
The second study conducted at Microsoft~\cite{contribution20} stated that code review do not find bugs. They found that only about 
15\% of the comments provided by reviewers indicate a possible defect, much less functionality issues that should block a 
code submission. 

Another empirical study on the effectiveness of security code review~\cite{contribution1}, conducted an experiment on 30 developers. They 
conducted manual code review of a small web application. The web application supplied to the developers had seven known vulnerabilities. 
Their findings concluded that none of the subjects found all confirmed vulnerabilities (they were able to find only 5 out of 7) 
and that reports of false vulnerabilities were significantly correlated with reports of valid vulnerabilities.

A different experiment argued that in large scale software programs developers spend much time to 
identify the bug reports (mainly due to the excessive number of duplicate bug reports)~\cite{contribution10}.

Keeping in mind the above discussion we did this questions:

\begin{enumerate}
  \item Are all the code review tools performing a lame number of defects detection?
With the abundance of data coming from the engineering systems and having a diverse set of projects to observe~\cite{contribution12, contribution13}, we ask 
if there is any code review providing more value than the others?

  \item What is the actual amount of time that developers take in individuating defects?
\end{enumerate}


\textit{Work in Progress...}

TODO:

\begin{itemize}
  \item What are the next steps in your research?
  \item What is the intended impact of your research?
  \item What are the most important challenges and threats of your research?
\end{itemize}

%\subsection{Subsection Heading}
%\label{sec:3}
%Your text goes here.

%\begin{equation}
%\vec{a}\times\vec{b}=\vec{c}
%\end{equation}

%\subsubsection{Subsubsection Heading}
%See Sect.~\ref{sec:1}.

%\paragraph{Paragraph Heading} %
%Your text goes here.

%\subparagraph{Subparagraph Heading.} Your text goes here.%
%
%\index{paragraph}
% Use the \index{} command to code your index words
%
% For tables use
%
%\begin{table}
%\centering
%\caption{Please write your table caption here}
%\label{tab:1}       % Give a unique label
%
% For LaTeX tables use
%
%\begin{tabular}{lll}
%\hline\noalign{\smallskip}
%first & second & third  \\
%\noalign{\smallskip}\hline\noalign{\smallskip}
%number & number & number \\
%number & number & number \\
%\noalign{\smallskip}\hline
%\end{tabular}
%\end{table}
%
%
% For figures use
%
%\begin{figure}
%\centering
% Use the relevant command for your figure-insertion program
% to insert the figure file.
% For example, with the option graphicx use
%\includegraphics[height=4cm]{figure.eps}
%
%\caption{Please write your figure caption here}
%\label{fig:1}       % Give a unique label
%\end{figure}
%
% For built-in environments use
%
%\begin{theorem}
%Theorem text\footnote{Footnote} goes here.
%\end{theorem}
%
% or
%
%\begin{lemma}
%Lemma text goes here.
%\end{lemma}
%
%
% BibTeX users please use
% \bibliographystyle{}
% \bibliography{}
%
% Non-BibTeX users please follow the syntax
% the syntax of "referenc.tex" for your own citations
%%%%%%%%%%%%%%%%%%%%%%%% referenc.tex %%%%%%%%%%%%%%%%%%%%%%%%%%%%%%
% sample references
% "computer science"
%
% Use this file as a template for your own input.
%
%%%%%%%%%%%%%%%%%%%%%%%% Springer-Verlag %%%%%%%%%%%%%%%%%%%%%%%%%%

%
% BibTeX users please use
% \bibliographystyle{}
% \bibliography{}
%
% Non-BibTeX users please use
\begin{thebibliography}{99.}
%
% and use \bibitem to create references.
%
% Use the following syntax and markup for your references
%
% Contributed Works
\bibitem{contribution1} Edmundson, A., Holtkamp, B., Rivera, E., Finifter, M., Mettler, A., Wagner, D. (2013). 
An empirical study on the effectiveness of security code review. In Engineering Secure Software and Systems 
(pp. 197-212). Springer Berlin Heidelberg.

\bibitem{contribution9} Bacchelli Alberto, and Christian Bird. 
"Expectations, outcomes, and challenges of modern code review." 
Proceedings of the 2013 International Conference on Software Engineering. IEEE Press, 2013.

\bibitem{contribution20} Czerwonka Jacek, Michaela Greiler and Jack Tilford. Code Reviews Do Not Find Bugs. 
How the Current Code Review Best Practice Slows Us Down. 
Proceedings of the 2015 International Conference on Software Engineering. IEEE Publisher, 2015.

\bibitem{contribution10} Tao Zhang; Byungjeong Lee "A Bug Rule Based Technique with Feedback for Classifying Bug Reports",  
Computer and Information Technology (CIT), 2011 IEEE 11th International Conference on, On page(s): 336 - 343 

\bibitem{contribution11} IaaS cloud computing of Rackspace Cloud and NASA: https://it.wikipedia.org/wiki/OpenStack and 
https://www.openstack.org

\bibitem{contribution12} Bug Tracking for OpenStack. https://it.wikipedia.org/wiki/Launchpad

\bibitem{contribution13} Gerrit Code Review for OpenStack. //review.openstack.org/Documentation/intro-quick.html

\bibitem{contribution14} D. L. Parnas and M. Lawford. Inspection's role in software quality assurance. In 
Software, IEEE,  vol. 20, 2003. 

\bibitem{contribution15} M. E. Fagan. Design and Code inspections to reduce errors in program development. In IBM Systems Journal 15 
pp. 182-211, 1976. 

\bibitem{contribution17} P. M. Johnson, and D. Tjahjono. Does Every Inspection Really Need a Meeting? In Empirical Software Engineering, vol. 3, 
no. 1, pp. 9-35, 1998. 

\bibitem{contribution18} P. McCarthy, A. Porter, H. Siy et al. An experiment to assess cost-benefits of inspection meetings and their 
alternatives: a pilot study. In Proceedings of the 3rd International Symposium on Software Metrics: From Measurement to Empirical 
Results, 1996. 

\bibitem{contribution19} A. Porter, H. Siy, C. A. Toman et al. An experiment to assess the cost-benefits of code inspections in 
large scale software development. In SIGSOFT Softw. Eng. Notes, vol. 20, no. 4, pp. 92-103, 1995. 

\bibitem{contribution16} D. E. Perry, A. Porter, M. W. Wade. Reducing inspection interval in large-scale software development. In Software 
Engineering, I EEE Transactions on, vol. 28, no. 7, pp. 695-705, 2002. 

\bibitem{contribution2} Broy M (2002) Ayewah, N., Hovemeyer, D., Morgenthaler, J. D., Penix, J., Pugh, W. (2008). 
Using static analysis to find bugs. Software, IEEE, 25(5), 22-29.

\bibitem{contribution3} W. R. Bush, J. D. Pincus, D. J. Sielaff. “A static
analyzer for finding dynamic programming errors,” Softw.
Pract. Exper.
, vol. 30, no. 7, pp. 775–802, 2000.

\bibitem{contribution4} Hallem, D. Park, and D. Engler, “Uprooting software
defects at the source,”
Queue, vol. 1, no. 8, pp. 64–71, 2003.

\bibitem{contribution5} B. Chess and J. West,
Secure Programming with Static
Analysis, 1st ed.  Addison-Wesley Professional, Jul. 2007.

\bibitem{contribution6} N. Kennedy.  How google does web-based code reviews with mondrian.
http://www.test.org/doe/, Dec. 2006.

\bibitem{contribution7} A.  Tsotsis.  Meet  phabricator,  the  witty  code  review  tool  built  inside
facebook.  http://techcrunch.com/2011/08/07/oh-what-noble-scribe-hath-
penned-these-words/, Aug. 2006.

\bibitem{contribution8} Gerrit code review - https://www.gerritcodereview.com/ 

\bibitem{contribution21} J.  M.  Gonzalez-Barahona,  G.  Robles,  and  D.  Izquierdo-Cortazar.   The
metricsgrimoire  database  collection.   In
12th  Working  Conference  on
Mining Software Repositories (MSR)
, pages 478–481, May 2015

\bibitem{contribution22} http://activity.openstack.org/dash/browser/data/db/


\end{thebibliography}

%%%%%%%%%%%%%%%%%%%%%%%%%%%%%%%%%%%%%%%%%%%%%%%%%%%%%%%%%%%%%%%%%%%%%%

%%%%%%%%%%%%%%%%%%%%%%%%%%%%%%%%%%%%%%%%%%%%%%%%%%%%%%%%%%%%%%%%%%%%%%

\printindex
\end{document}





